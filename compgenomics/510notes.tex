\documentclass[font=12pt]{article}
\usepackage{../hdr}
\newcommand{\mycourse}{02-510 Computational genomics}

\begin{document}

\medskip

\section{Normalization}
\textcolor{red}{RPKM} is a very important normalization tool.\\
\textcolor{blue}{not too sure about tpm, need to double check on that}

\section{Single Cell}
Goals:
\begin{enumerate}
	\item cells differentiates into sub-celltypes
	\item unknown celltype discovery
\end{enumerate}

\subsection{Dimensionality reduction}
Motivation:
\begin{enumerate}
	\item high dim data often has lower dim representation w/o much reconstruction error
	\item lower dim representation can often represent info about high dim pairwise dist.
\end{enumerate}
Types of dim-red:
\begin{itemize}
	\item \textcolor{red}{Global methods}
		\begin{enumerate}
			\item all pairwise dist equally impt
			\item lower dim pairwise dist fit high-dim ones
			\item often use magnitude or rank order
		\end{enumerate}
	\item \textcolor{red}{Local methods}
		\begin{enumerate}
			\item only local dist reliable in high dim
			\item more weight on modelling local dist correctly
		\end{enumerate}
\end{itemize}
Methods:
\begin{itemize}
	\item PCA
		\begin{itemize}
			\item finds directions with largest variance
			\item minimize squared reconstruction error
			\item equiv to liner autoencoders
			\item Steps of PCA
			\begin{enumerate}
				\item $\overline{X}$: mean of all samples(usually rows), adjust $ X \rightarrow X' = X - \overline{X}$ 
				\item covar matrix $ C = X'^T X'$
				\item find eigenvectors and eigenvalues of $C$, ie. all pair of $\vec{v}, \lambda$ st. $ C\vec{v} = \lambda \vec{v}$
				\item eigenvalues can be used to calculate percentage of total variance for each component \[v_j = 100\frac{\lambda_j}{total \quad eigenvalue}\] 
			\end{enumerate}
		\end{itemize}
		\textcolor{blue}{This is non-parametric method, do not insist on a parametric encoding function}
	\item Multi-Dimensional Scaling
		\begin{itemize}
			\item  arrange low dim points to minimize diff between pairwise distances in the high and low D space
			\item a possible approach: start w a random vector, perform gradient decent
			\item \textcolor{red}{is there something to do with PCA?} then we don't need iterative method
		\end{itemize}
	\item Sammon (non-linear autoencoder)
		\begin{itemize}
			\item  with extra layers, much more powerful than PCA, but can be slow to converge, and can get stuck on local optima
			\item Multi-Dimensional Scaling(MDS) can be made non-linear by giving higher weights to smaller distances, a popular formula is 
				\[ cost = \sum_{ij} \left(\frac{|x_i - x_j| - |y_{i}-y_j|}{|x_i - x_j|}\right)^2\]
				where $ x$ is high-dim dist, and  $y$ is low-dim dist
			\item still slow and get stuck on local optima
		\end{itemize}
\end{itemize}

\subsection{Graph-basd method}
		\begin{itemize}
			\item address uniform circularity
			\item  \textcolor{red}{Isomap} is a dim-red technique based on graphs
				\begin{itemize}
					
			\item each datapoint is connected to $k$ nearest neighbor in high-dim
			\item edge weights = euclidean dist
			\item approx of distance = shortest path in contracted graph
			\end{itemize}
		\item \textcolor{red}{Probabilistic local MDS}
			\begin{itemize}
				\item local distances are more impt than non-local ones
				\item in this way all local distances are given equal importance	\end{itemize}

		\item \textcolor{red}{stochastic neighbor embedding(SNE)} has a probabilistic way to decide if a distance is local
			\begin{itemize}
				\item convert global distances into probability of one datapoint picking another datapoint as its neighbor \textcolor{blue}{(what defines a neighbor tho) - still about isomaps?}
				\item each point in high-dim has a conditional probability of picking any other point as its neighbor
			\item distribution (some sort of Gaussian) is over high-dim distances (if high-dim coords unavailable, a similarity / dissimilarity matrix may be used)\\ $p_{j|i}$ is the prob. of picking j given starting at i in high-dim.
				\[ P_{j|i} = \frac{e^{-2d_{ij}^2 / 2\sigma_i^2}}{\sum_k e^{{-d_{ik}^2}/{2\sigma_i^2}}} \]
				\item having the probabilities potentially allow us to throw away the raw high-dimensional data
				\item evaluation done using pairwise distance in low dimensional map (shows how well the lower dim representation models high-dim data ig)\\
					$q_{j|i}$ is the prob of picking j given starting at i in low-dim
				\item compute the \textcolor{red}{Kullback-Leibler divergence} between prob. in the high-dim and low-dim spaces \textcolor{blue}{(why not just use dist in high dim?) - more space / time efficient}
				\item nearby pts in high-dim should be close in low-dim
			\end{itemize}
		\item picking $\sigma$ used to compute  $p$ - the radius of the gaussian
			 \begin{itemize}
				 \item different radius is needed in different parts of the space to keep the no. of neighbors constant
				 \item big radius $\rightarrow$ high entropy for distribution over i's neighbors
				 \item small radius $\rightarrow$ low entropy
				
			\end{itemize}
		\item \textcolor{red}{Symmetric SNE}
			\begin{itemize}
				\item simpler than stochastic
				\item works best if different procedures are used for computing $p$'s and  $q$'s.
				\item compromise: no longer guarantees that if using same dimension will produce optimal solution
				\item turn conditional prob into symmetric pairwise probabilities
			\end{itemize}
		\item Optimization methods for SNE
			\begin{itemize}
				\item simulated annealing could lead to better global optimization
				\item add Gaussian noise to $y$ location in each update
				\item spend longer time at noise level where global structure start to form
				\item \textcolor{red}{t-SNE} - use Gaussian at many (infinite) spatial scales, cheaper as we don't have to exponentiate anymore \textcolor{blue}{(why????)} 
			\end{itemize}
				\end{itemize}
\subsection{Supervised dim-red: Neural networks}
\begin{itemize}
	\item last few layers have much fewer values than inputs
	\item use intermediate layers as lower-dim representations
	\item can easily add prior biological knowledge, such as protein interactions or transcription factors
	\item essentially, some nodes in the hidden layer are same as before, others are based on biological info
\end{itemize}
\subsubsection{Additional NN architecture: Siamese}
\begin{itemize}
	\item supervised, but not trying to maximize training accuracy
	\item input: whether each pair is similar
	\item output: binary label of similar / not similar
	\item thus directly optimize dim-red layer for KNN
\end{itemize}

\section{ChIP-Seq}

\section{cis-regulatory motif}
\end{document}
